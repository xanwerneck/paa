\documentclass[a4paper, 11pt]{article}

\usepackage[utf8]{inputenc}
\usepackage[brazil]{babel}
\usepackage[T1]{fontenc}
\usepackage{lmodern}
\usepackage{amsmath}
\usepackage{float}
\usepackage{graphicx}
\usepackage{epstopdf}
\usepackage{caption}
\usepackage{subcaption}

\setcounter{section}{-1}

\DeclareGraphicsExtensions{.eps}

\title{Relatório -- Implementation Work}
\author{Alexandre Werneck\\Gabriel de Quadros Ligneul}
\date{12 de Dezembro de 2016}

\begin{document}

\maketitle

\section{Introdução}

TODO.

\section{On the Fractional Knapsack Problem}

\subsection{Pseudocódigo}

TODO.

\subsubsection{Prova que o Algoritmo é Ótimo}

TODO.

\subsection{Avaliação da Complexidade}

TODO.

\subsection{Testes de Desempenho}

\begin{figure}[H]
  \centering
  \includegraphics[width=\linewidth]{plots/matmul2}
  \caption{TODO.}
  \label{fig:matmul2}
\end{figure}

\section{Moving on Towards Integer Optimization}

\section{Adding Some Spice to the Problem}


\end{document}

