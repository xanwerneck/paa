\documentclass[a4paper, 11pt]{article}

\usepackage[utf8]{inputenc}
\usepackage[brazil]{babel}
\usepackage[T1]{fontenc}
\usepackage{lmodern}
\usepackage{amsmath}
\usepackage{float}
\usepackage{graphicx}
\usepackage{epstopdf}
\usepackage{caption}
\usepackage{subcaption}
\usepackage{algorithm}
\usepackage{algpseudocode}
\usepackage[a4paper,left=3cm,right=3cm,top=2.5cm, bottom=2.5cm]{geometry}
\usepackage{amsthm}

\newtheorem{lemma}{Lema}

\setcounter{section}{-1}

\DeclareGraphicsExtensions{.eps}

\makeatletter
\renewcommand{\ALG@beginalgorithmic}{\scriptsize}
\renewcommand{\ALG@name}{Algoritmo}
\makeatother

\algtext*{EndFor}
\algtext*{EndIf}
\algtext*{EndWhile}
\algtext*{EndFunction}
\algrenewcommand{\algorithmiccomment}[1]{// #1}
\newcommand{\algsection}[2]{(linha \ref{alg:#1:#2:start} à \ref{alg:#1:#2:end})}

\title{Projeto e Análise de Algoritmos\\Relatório -- Trabalho de Implementação}
\author{Alexandre Werneck\\Gabriel de Quadros Ligneul}
\date{12 de Dezembro de 2016}

\begin{document}

\maketitle

%-------------------------------------------------------------------------------
\section{Introdução}

A implementação dos algoritmos propostos neste trabalho foram feitos na linguagem Python.
Para executar o programa implementado, basta rodar a linha de comando \texttt{python3 knapsack.py <arquivo de entrada> <questão>}, onde a questão pode ser 1, 2 ou 3.

Utilizamos uma estrutura de dados com peso, valor e lista de conflitos para representar cada item que faz parte da entrada.
No nosso programa em Python, utilizamos classes para representar tal estrutura.
A escrita e o acesso a cada campo da classe é feita em tempo constante.
As listas utilizadas nos algoritmos são implementadas como \emph{arrays} e possuem tempo de acesso constante com base nos índices.

Todos os testes de desempenho foram feitos no mesmo computador, que possui um processador Intel Core i7-4770 e sistema operacional Linux.
Medimos o tempo real (\emph{wall-clock time}) da execução usando o programa \texttt{time}.
Para cada instancia do problema, foram realizadas 10 medições e apenas o menor tempo foi considerado.

%-------------------------------------------------------------------------------
\section{Problema da Mochila com Frações}

\subsection{Pseudocódigo}

Propomos o algoritmo \ref{alg:q1} para obter a solução ótima para esta versão do problema.

\begin{algorithm}[htb]
\begin{algorithmic}[1]
  \Function{FractionalKnapsack}{$itens, n, W$}
    \For{$i \gets 1, n$}
      \State {$itens[i].ratio \gets itens[i].weight / itens[i].value$}
    \EndFor
    \State {Order $itens$ using the $ratio$ field}
    \State $selected \gets $ empty list
    \State $w \gets 0$
    \For{$i \gets 1, n$}
      \If{$w \ge W$}
        \State \textbf{break}
      \ElsIf{$w + itens[i].weight \le W$}
        \State $frac \gets 1$
      \Else
        \State $frac \gets (W - w) / item[i].weight$
      \EndIf
      \State $selected.append(\{i, frac\})$
    \EndFor
    \State \Return selected
  \EndFunction
\end{algorithmic}
\caption{Resolução do problema da mochila com frações.}
\label{alg:q1}
\end{algorithm}

\subsubsection{Prova que o Algoritmo é Ótimo}

Dado o problema da mochila fracionária representado por $P$, um conjunto de itens $I = \{i_1,\ldots,i_n\}$ e uma capacidade $W$ da mochila, sabemos que $P$ admite uma solução ótima dada por $O = \{o_1,\ldots,o_k\}$.
Existe ainda uma solução $G = \{g_1,\ldots,g_j\}$ obtida através da execução do algoritmo guloso.
$G$, portanto, segue o critério de ordenação descendente dos itens em função da razão $R(i)$, calculada dividindo o valor do item pelo seu peso.
Sendo assim, $R(g_i) \ge R(g_{i+1})$ para todo $i = 1\ldots j-1$.

Queremos provar que a solução do algoritmo guloso $G$ faz parte do conjunto das soluções ótimas do problema $P$.

\begin{lemma}
O item $g_1$ da solução gulosa pertence a alguma solução ótima $O'$.

Caso 1: $g_1$ é adicionado por inteiro.
Se $g_1$ pertence a $O$, $O' = O$. Se o item não pertencer a solução $O$, podemos retirar de $O$ frações que somem o mesmo peso de $g_1$ e então inserir $g_1$.
Dessa forma, teremos uma solução ótima $O'$ que possui o mesmo valor de $O' = O \cup \{g_1\}$.
É trivial observar que se $O$ não possuir frações de itens com mesma razão que $g_1$, $O$ não seria uma solução ótima.

Caso 2: $g_1$ é adicionado de forma fracionária, onde $f_1 =$ fração de $g_1$.
Se $f_1 \cdot peso(g_1)$ pertence a $O$, $O' = O$. 
Se $O$ não contém $f_1 \cdot peso(g_1)$, assim como o item anterior, podemos trocar o equivalente de $f_1 \cdot peso(g_1)$ em $O$ e obter uma solução ótima $O'$.
\end{lemma}

\begin{lemma}
Agora queremos provar que a solução $O' - {g_1}$ é igual a solução de um novo problema dado por $P'$ que contém os itens $I' = I - {g_1}$ e capacidade da mochila dada por $W' = W - peso(g_1)$.

Vamos provar por contradição que $O'' = O' - {g1}$, no qual $O'$ corresponde às soluções obtidas no lema 1.

Assumindo que $O''$ não é solução ótima para o problema $P'$. Então $P'$ admite uma solução $Y > O''$. Seja $X = Y \cup \{g_1\}$, e $O'$ possui valor igual a $O'' \cup \{g_1\}$. Dessa forma teremos em $X > O'$ uma contradição, dado que O' é uma solução ótima.
\end{lemma}

De forma iterativa e com base no Lema 2, mostramos que o resultado do algoritmo guloso é uma solução ótima para a versão fracionária do problema da mochila.

\subsection{Avaliação da Complexidade}

A complexidade do algoritmo proposto pode ser distribuída em dois momentos.
O primeiro deles é referente à ordenação dos itens em função da razão entre o peso e o valor de cada item candidato.
Sabemos que o problema de ordenação pode ser resolvido em $\theta(nlogn)$, logo essa é a complexidade dessa primeira parte do algoritmo.

A segunda etapa consiste em iterar pela lista ordenada e a cada passo adicionar o item com \emph{menor} razão.
Dentro do laço de iteração são tratados os casos que garantem que a soma dos itens selecionados não ultrapasse a capacidade da mochila.
Como são apenas realizadas operações constantes para cada passo do \emph{loop}, a complexidade dessa parte é $\theta(n)$.

Sendo assim, o algoritmo possui complexidade $\theta(n  + nlogn)$ e, portanto, pode ser resolvido em $\theta(nlogn)$.

\subsection{Testes de Desempenho}

A tabela \ref{tab:q1:bench} apresenta os resultados obtidos para a implementação da primeira questão.
Como era esperado, a variação do peso aceito pela mochila não influencia na complexidade do algoritmo.
Com base na figura \ref{fig:q1:itens}, observamos que a relação entre o número de itens e o tempo de execução é próxima a função $nlog(n)$.
Dessa forma, concluímos que os resultados obtidos estão de acordo com a complexidade avaliada no item anterior.

\begin{table}[htb]
\centering
\begin{tabular}{c|c|c|c|c}
Itens & W=1000 & W=2000 & W=5000 & W=10000 \\
\hline
120 & 0.020s    & 0.020s    & 0.020s    & 0.020s    \\
250 & 0.021s    & 0.021s    & 0.021s    & 0.021s    \\
500 & 0.024s    & 0.024s    & 0.024s    & 0.025s    \\
1000 & 0.035s   & 0.035s    & 0.035s    & 0.035s    \\
\end{tabular}
\caption{Medida do tempo de execução (em segundos) da implementação da questão 1 para as instâncias disponibilizadas.}
\label{tab:q1:bench}
\end{table}

\begin{figure}[htb]
\centering
\includegraphics[width=0.5\linewidth]{plots/q1itens}
\caption{Variação do tempo de execução em relação ao número de itens.}
\label{fig:q1:itens}
\end{figure}

%-------------------------------------------------------------------------------
\section{Problema da Mochila com Inteiros}

\subsection{Contraexemplo para o Algoritmo Guloso}

O algoritmo da questão anterior não retorna a solução ótima em todos os casos para esta versão do problema. Um contraexemplo é uma instância com dois itens: o item 1 com peso 10kg e valor \$10 e o item 2 com peso 1kg e valor \$2. Numa mochila que aguenta 10kg, o algoritmo anterior escolheria primeiro o item 2, pois ele tem a maior razão entre valor e peso. Não seria possível inserir o item 1 no conjunto final da solução, pois ultrapassaria o limite da mochila, e consequentemente o lucro seria \$2. Entretanto, a solução ótima para essa instância do problema contém apenas o item 1, possibilitando \$10 de lucro.

\subsection{Algoritmo Ótimo}

Propomos o algoritmo \ref{alg:q2}, que utiliza programação dinâmica, para obter a solução ótima para este problema.

\begin{algorithm}[htb]
\begin{algorithmic}[1]
  \Function{IntegerKnapsack}{$itens, n, W$}
    \State \Comment{Initialization:} \label{alg:q2:init:start}
    \State $M \gets zeroed matrix(n, W)$ \Comment{Optimum profit}
    \State $S \gets zeroed matrix(n, W)$ \Comment{Selected itens} \label{alg:q2:init:end}
    \State \Comment{Find the best solution for each subproblem:} \label{alg:q2:calcopt:start}
    \For{$i \gets 1, n$}
      \For{$w \gets 1, W$}
        \State $li \gets M[i - 1, w]$ \Comment{Best solution for last item}
        \If{$itens[i].weight > w$}
          \State $M[i, w] \gets li$
        \Else
          \State \Comment{$ti$ is the best solution with this item:}
          \State $ti \gets itens[i].value + M[i- 1, w - itens[i].weight]$
          \If{ti > li}
            \State $M[i, w] \gets ti$
            \State $S[i, w] \gets 1$ \Comment{Use this item in the solution}
          \Else
            \State $M[i, w] \gets li$
          \EndIf
        \EndIf
      \EndFor
    \EndFor \label{alg:q2:calcopt:end}
    \State \Comment{Build the list of selected itens:} \label{alg:q2:select:start}
    \State $selected \gets $ empty list
    \State $w \gets M[n, W]$
    \For{$i \gets n, 1$}
      \If{$S[i, w] == 1$}
        \State $selected.append(i)$
        \State $w \gets w - itens[i].weight$
      \EndIf
    \EndFor
    \State \Return selected \label{alg:q2:select:end}
  \EndFunction
\end{algorithmic}
\caption{Resolução do problema da mochila com valores inteiros.}
\label{alg:q2}
\end{algorithm}

\subsubsection{Avaliação da Complexidade}

O primeiro trecho do algoritmo \algsection{q2}{init} cria duas matrizes de tamanho $n \times W$, percorre todas as suas posições e as inicializa com 0. Entretanto, como o tamanho da matriz varia de acordo com o \emph{valor} de $W$, não podemos avaliar a complexidade como $\theta(nW)$. Definimos K como o tamanho (número de bits) de $W$ e sabemos que $W \approx 2^K$. Logo, levando em conta o \emph{tamanho} de $W$, a complexidade desse trecho é de $\theta(n2^K)$. O segundo trecho \algsection{q2}{calcopt} realiza uma série de operações em tempo constante para cada posição da matriz $M$, logo a complexidade desse último também é $\theta(n2^K)$. Por fim, o terceiro trecho \algsection{q2}{select} percorre a lista de itens e insere (operação constante) os selecionadas na lista de saída. No pior caso, todos os itens são selecionados e toda a lista de entrada é percorrida, logo avaliamos a complexidade como $\theta(n)$. A complexidade da função é a soma da complexidade dos três trechos: $T(n, K) \in \theta(n2^K + n2^K + n) \in \theta(n2^K)$.

\subsection{Avaliação do Desempenho}

A tabela \ref{tab:q2:bench} apresenta o tempo de execução do programa implementado para as instâncias fornecidas.
Com base no gráfico da esquerda na figura \ref{fig:q2:itens}, podemos confirmar que o tempo de execução cresce linearmente em relação ao número de itens. Igualmente, a partir do gráfico da direita, observamos que o tempo de execução cresce linearmente em relação à capacidade da mochila. Dessa forma, o tempo de execução cresce exponencialmente em relação ao tamanho K (número de bits) desse peso.

\begin{table}[htb]
\centering
\begin{tabular}{c|c|c|c|c}
Itens & W=1000 & W=2000 & W=5000 & W=10000 \\
\hline
120 & 0.069s& 0.123s& 0.290s& 0.594s \\
250 & 0.123s& 0.234s& 0.578s& 1.184s \\
500 & 0.241s& 0.476s& 1.196s& 2.397s \\
1000 & 0.482s& 0.963s& 2.432s& 4.911s \\
\end{tabular}
\caption{Medida do tempo de execução (em segundos) da implementação da questão 2 para as instâncias disponibilizadas.}
\label{tab:q2:bench}
\end{table}

\begin{figure}[htb]
\centering
\includegraphics[width=0.49\linewidth]{plots/q2itens}
\includegraphics[width=0.49\linewidth]{plots/q2weight}
\caption{Variação do tempo de execução em relação ao número de itens e ao peso aceito pela mochila.}
\label{fig:q2:itens}
\end{figure}

%-------------------------------------------------------------------------------
\section{Problema da Mochila com Conflitos}

\subsection{Algoritmo Proposto}

A proposta que trazemos para a resolução deste problema consiste em usar diferentes heurísticas e retornar o melhor resultado. Sendo assim, é possível que o algoritmo utilize uma heurística diferente a cada entrada. As quatro heurísticas que usamos são apresentadas na tabela \ref{tab:q3:heuristics}. A partir daí a solução segue a estrutura do algoritmo guloso da questão 1, em que os itens são adicionados na mochila com as restrições que o problema apresenta. A cada item a ser adicionado são verificados se ao adicioná-lo o peso da mochila excede o limite e se o mesmo não possui conflito com os itens já adicionados. O algoritmo \ref{alg:q3} descreve detalhadamente tal solução.

\begin{table}[htb]
\centering
\renewcommand{\arraystretch}{2}
\begin{tabular}{c|c|c|c}
H1 & H2 & H3 & H4 \\\hline
$\dfrac{peso}{valor}$ &
$\dfrac{1}{valor}$ &
$\sum^{conflitos}_{c} c.valor$ &
$\dfrac{1}{\sum^{conflitos}_{c}\dfrac{c.peso}{c.valor}}$ \\
\end{tabular}
\caption{Heurísticas utilizadas para resolver o problema da mochila com conflitos.}
\label{tab:q3:heuristics}
\end{table}

\begin{algorithm}[htb]
\begin{algorithmic}[1]
  \State \Comment{Obtains the group of itens using the heuristic H:}
  \Function{ConflictsKnapsackHelper}{$itens, n, W, H$}
    \For{$i \gets 1, n$} \label{alg:q3:heuristics:start}
      \State {$itens[i].heuristic \gets H(itens[i])$}
    \EndFor \label{alg:q3:heuristics:end}
    \State {Order $itens$ using the $heuristic$ field} \label{alg:q3:sort}
    \State $hasconflict \gets $empty list \label{alg:q3:init:start}
    \For{$i \gets 1, n$}
      \State $hasconflict[i] \gets False$
    \EndFor
    \State $selected \gets $ empty list
    \State $w \gets 0$ \label{alg:q3:init:end}
    \For{$i \gets 1, n$} \label{alg:q3:select:start}
      \If{$!hasconflict[itens[i].index] \textrm{ \textbf{and} } itens[i].weight + w \le W$}
        \State $selected.append(i)$
        \For{$c \textrm{ \textbf{in} } itens[i].conflicts$}
          \State $hasconflict[c.index] \gets True$
        \EndFor
      \EndIf
    \EndFor \label{alg:q3:select:end}
    \State \Return selected
  \EndFunction
  \State \Comment{Return the most valuable set of itens given the heuristics:}
  \Function{ConflictsKnapsack}{$itens, n, W$}
    \State $bestsack \gets Nil$
    \State $bestvalue \gets 0$
    \For{$H \textrm{ \textbf{in} } heuristics$}
      \State $sack \gets ConflictsKnapsackHelper(itens, n, W, H)$
      \State $value \gets ComputeSackValue(sack)$ \label{alg:q3:sackval}
      \If{$value \ge bestvalue$}
        \State $bestsack \gets sack$
        \State $bestvalue \gets value$
      \EndIf
    \EndFor
    \State \Return $bestsack$
  \EndFunction
\end{algorithmic}
\caption{Resolução do problema da mochila com conflitos.}
\label{alg:q3}
\end{algorithm}

\subsection{Avaliação da Complexidade}

Definimos $T_h(n)$ como a complexidade da função auxiliar \emph{ConflictsKnapsackHelper}. A complexidade para o cálculo da heurística \algsection{q3}{heuristics} é feito em $\theta(n)$ para H1 e H2, pois para cada item são apenas feitas operações constantes. Já para as heurísticas H3 e H4, são realizadas uma série de operações para cada conflito de cada item. Dessa forma, sabemos que a complexidade desse trecho é $\theta(\sum_{i}^{itens}|i.conflicts|) \in \theta(n + m)$, onde m é o número total de conflitos. A linha \ref{alg:q3:sort} realiza a ordenação da lista de itens, que pode ser feita em $\theta(nlogn)$. Em seguida, é trivial observar que as estruturas auxiliares \algsection{q3}{init} são inicializadas em tempo linear em relação ao número de itens. Por fim \algsection{q3}{select}, a lista de itens e os conflitos de cada um são percorridos, logo a complexidade é $\theta(n + m)$. Somando a complexidade de todos os trechos da função, obtemos $T_h(n) \in \theta(nlogn + n + m) \in \theta(nlogn + m)$.

A função \emph{ConflictsKnapsack} apenas chama a função \emph{ConflictsKnapsackHelper} 4 vezes e seleciona qual conjunto de itens provém o maior lucro. Na linha \ref{alg:q3:sackval}, o lucro do conjunto de itens resultante é calculado em tempo linear em relação ao número de itens da entrada. Logo, a complexidade da função \emph{ConflictsKnapsack} é $T(n) \in \theta(4(n + T_h(n))) \in \theta(4(n + nlogn + m)) \in \theta(nlogn + m)$.

\subsection{Avaliação do Desempenho}

A tabela \ref{tab:q3:bench} apresenta os resultados obtidos ao rodar as instâncias providas para o problema 3.
Como era esperado, notamos que o tamanho aceito pela mochila não influencia na complexidade do algoritmo.
Já com base na figura \ref{fig:q3:itens}, podemos observar que o tempo de execução é de fato linear em relação ao número de conflitos.
Por fim, dado que o número de conflitos nas instâncias é bem maior que o número de itens, acreditamos que este último não seja significativo para nossas comparações.

\begin{table}[htb]
\centering
\begin{tabular}{c|c|c|c|c|c}
Itens & Conflitos & W=1000 & W=2000 & W=5000 & W=10000 \\
\hline
120 & 3570 & 0.023s & 0.023s    & 0.022s    & 0.023s    \\
250 & 15562 & 0.034s    & 0.034s    & 0.034s    & 0.034s    \\
500 & 62375 & 0.079s    & 0.079s    & 0.079s    & 0.079s    \\
1000 & 249750 & 0.324s  & 0.321s    & 0.325s    & 0.325s    \\
\end{tabular}
\caption{Medida do tempo de execução (em segundos) da implementação da questão 3 para as instâncias disponibilizadas.}
\label{tab:q3:bench}
\end{table}

\begin{figure}[htb]
\centering
\includegraphics[width=0.5\linewidth]{plots/q3itens}
\caption{Variação do tempo de execução em relação ao número de conflitos.}
\label{fig:q3:itens}
\end{figure}

\end{document}
